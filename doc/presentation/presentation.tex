\documentclass{beamer}
\author{Julius Elinson}
\usetheme{Frankfurt}
\usecolortheme{beaver}
\usepackage{multicol, amsmath,mathpazo}
\title{Image-Based Rock-Climbing Simulator}
\subtitle{Artificial Intelligence Final Project}
\institute{Harvey Mudd College \\CS 151}
\date{\today}

\beamertemplatenavigationsymbolsempty

\begin{document}
\frame{\titlepage}


\section{Problem}
\begin{frame}{Rock Climbing}
\begin{columns}

\begin{column}[T]{.5\linewidth}
\includegraphics[height=\linewidth,angle=-90]{img/test2.JPG}
\end{column}
\pause
\begin{column}[T]{.6\linewidth}
\vspace{.8cm}
Problem Definition
\begin{itemize}
 \item Routes are color-delimited
 \item Use any subset of designated grips to get to the top
 \item Difficulty determined by size, spacing and surface properties of the grips
 \item Climbers have to determine a path
\end{itemize}

\end{column}

\end{columns}
\end{frame}


\begin{frame}{Rock Climbing}
\begin{columns}

\begin{column}[T]{.5\linewidth}
\includegraphics[height=\linewidth,angle=-90]{img/test2.JPG}
\end{column}
\begin{column}[T]{.6\linewidth}
\vspace{.8cm}
Challenges
\pause
\begin{itemize}
 \item Physically-viable arrangements
 \item Balance forces
 \item Minimize muscle strain
 \item Path efficiency
\end{itemize}
\pause
\vspace{.3cm}
Constraints
\begin{itemize}
 \item Motion in 2D plane -- no overhang
 \item Depth correlated to grip size
\end{itemize}
\end{column}

\end{columns}
\end{frame}

\section{Approach Overview}

\begin{frame}[t]{Solution Specifications}
Input:
\begin{itemize}
 \item A low-resolution color photo of a rock wall
 \item Single pixel selection by user that maps to one of the grips in the desired route
\end{itemize}
\vspace{.3cm}
\pause

Output:
\begin{itemize}
 \item A viable path that minimizes a specified cost function
 \item Rendering of climber positions along the solution path
 \item Strain analysis
\end{itemize}

\pause
\vspace{.3cm}
Tools:
\begin{itemize}
 \item OpenCV Library
 \item Qt C++ Framework
\end{itemize}
\end{frame}

\begin{frame}[T]{Pipeline}

\pause

\begin{block}{Image Processing}
\begin{itemize}
 \item Route Detection
 \item Grip Analysis
\end{itemize}
\end{block}

\pause

\begin{block}{Heuristic Analysis}
\begin{itemize}
 \item Modeling Grip Support
\end{itemize}
\end{block}

\pause

\begin{block}{Physics Engine}
\begin{itemize}
 \item Modeling a Human
 \item Simulation Motion
\end{itemize}
\end{block}

\pause

\begin{block}{Path Search}
\begin{itemize}
 \item Starting a Route
 \item BFS \& A$^*$
\end{itemize}
\end{block}

\end{frame}


\section{Image Processing}


\begin{frame}{Image Processing}
\begin{columns}
\begin{column}[T]{.5\linewidth}
\includegraphics<1>[width=\linewidth]{img/test2c.jpg}
\includegraphics<2>[width=\linewidth]{img/hue.jpg}
\includegraphics<3>[width=\linewidth]{img/noisy.jpg}
\includegraphics<4>[width=\linewidth]{img/threshold.jpg}
\includegraphics<5>[width=\linewidth]{img/contour.jpg}
\end{column}


\begin{column}[T]{.5\linewidth}
Steps:
\begin{itemize}
 \pause
 \item Convert to HSV color space
 \pause
 \item Threshold image by hue of user selection
 \pause
 \item Denoise image using dilation \& erosion
 \pause
 \item Represent grips as contour
\end{itemize}
\end{column}

\end{columns}
\end{frame}

\begin{frame}[t]{Grip Analysis}
\begin{center}
\includegraphics[width=.45\linewidth]{img/close_up.jpg}  \hspace{.05cm}
\includegraphics[width=.455\linewidth]{img/close_up_c.jpg}
\end{center}
\pause
Physical Properties:\vspace{-.15cm}
\begin{multicols}{2}
\begin{itemize}
 \item[-] Area
 \item[-] Perimeter
 \item[-] Center of Mass
 \item[-] Convexity Defects
\end{itemize}

\end{multicols}
\end{frame}


\begin{frame}[t]{Grip Analysis}
Orientation Estimation
\begin{center}
\includegraphics[width=.45\linewidth]{img/pre_normal.jpg}  \hspace{.05cm}
\pause
\includegraphics[width=.45\linewidth]{img/normal_field.jpg}
\end{center}

\end{frame}



\section{Heuristic Analysis}
\begin{frame}[t]{Grip Heuristics}
\pause
Binary Criteria
\begin{itemize}
 \item Can it support a hand or just a foot?
 \item Can it support two limbs?
\end{itemize}
\pause
\vspace{.3cm}
Continuous Variables
$$
F = f(a, p, d, N, \theta)
$$
where $a$ is area, $p$ is perimeter, $d$ are the convexity defects, $N$ is the normal field, and $\theta$ is the angle at which the grip is grabbed.\\
\pause
As an approximation, 
$$
F = a \cdot N[\theta] + \text{hardlim}(d) \cdot |d|.
$$
\end{frame}

\section{Physics Engine}
\begin{frame}
 crap
\end{frame}

\section{Path Search}
\begin{frame}
 crap
\end{frame}


\section{Progress}
\begin{frame}
 crap
\end{frame}

\end{document}